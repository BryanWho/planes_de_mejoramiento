\documentclass[spanish,12pt,a4paper]{article}
\usepackage[T1]{fontenc}
\usepackage{geometry}
\usepackage{graphicx}
\usepackage{mathtools}
\usepackage{amssymb}
\usepackage{amsthm}
\usepackage{xcolor}
\usepackage{babel}
\usepackage{hyperref}
\usepackage{longtable}
\title{Informe comisión y promoción}
\author{Brayan Romero}
\begin{document}
	
	%
	\begin{center}
		\begin{tabular}{|p{3cm}|p{12cm}|}
			\hline
			\vspace{0.05cm}
			\centering
			\includegraphics[scale=0.35]{logo.png}
			&
			\begin{center}
				\textbf{GIMNASIO PSICOPEDAGÓGICO SUBA}
				
				\vspace{0.1cm}
				El amor, fuente fundamental para la formación integral
				
				\vspace{0.1cm}
				\textbf{PLAN DE MEJORAMIENTO}
				
				\vspace{0.1cm}
				\textbf{GRADO NOVENO}
				
			\end{center}
			\\
			\hline
			
			
		\end{tabular}
		
	\end{center}	
	
	\section*{OBJETIVO:}
	Resolver problemas en los cuales se hace necesaria la utilización de operaciones aritméticas entre números naturales para aplicarlo en contextos reales de su vida cotidiana. 
	
	\section*{PROCEDIMIENTO Y CRONOGRAMA:}
	\begin{enumerate}
		\item Entrega del plan de mejoramiento al padre de familia.
		\item Entrega de los talleres desarrollados al docente de la asignatura, el día \textcolor{red}{26 de agosto de 2024} en físico.
		\item El estudiante debe realizar la sustentación el día \textcolor{red}{26 de agosto de 2024}.
	\end{enumerate}
	
	\section*{OBSERVACIONES:}
	\textbf{Se solicita leer las observaciones y estudiar la actividad antes de comenzar.}
	\begin{itemize}
		\item Si tiene alguna duda sobre los puntos del plan, diríjase al docente para aclararlas, en horas laborales, antes de entregar el plan de mejoramiento, de no hacerlos, no se aceptarán reclamos que estén fuera de los lineamientos ni excusas.
		
		\item Todas las actividades asignadas son de carácter obligatorio, ya que hacen parte de los temas trabajados durante el periodo. Estas deben ser entregadas sin excepción y sustentadas según fecha indicada.
		
		\item La presentación del trabajo completo, en la fecha correspondiente, es indispensable para la evaluación de la misma.
		
		\item La presentación completa del trabajo tendrá un valor del 40\% y su evaluación respectiva 60\%. Toda asignatura recuperada tendrá una nota de Aprobado (DB) (70): Desempeño Básico; y su nota será modificada en el boletín correspondiente al siguiente período.
		
		\item \textbf{Por lo anterior se debe tener en cuenta que el estudiante que no pase la evaluación tampoco pasará la recuperación. Por lo mismo, aquel que no presente alguna actividad tampoco podrá presentar la evaluación.}
		
		\item Los estudiantes deberán asistir a las actividades de recuperación con el uniforme.
		
		\item La actividad debe ser desarrollada en su totalidad por el ESTUDIANTE a mano.
		
		\item La presentación del trabajo se tendrá en cuenta. Trabajos arrugados y hechos a la carrera no serán calificados. \textbf{La caligrafía y ortografía} debe ser digna de un trabajo de recuperación. Se recomienda una excelente presentación de los trabajos con portada y en carpeta que sean dignos de calificación.
		
		\item \textbf{El orden, la actitud y la disciplina al momento de la sustentación también se tendrá en cuenta.}
		
		\item \textbf{Respuesta copiada textual de internet, será anulada.}
	\end{itemize}
	
	\vspace{0.5cm}
	\section*{PLAN DE MEJORAMIENTO}
	\subsection*{Grado Noveno}
	
	A continuación se presenta la lista de desempeños a evaluar en el plan de mejoramiento, cada estudiante realizará los puntos según el desempeño perdido. El estudiante que tenga la asignatura reprobada deberá realizar el plan de mejoramiento en su totalidad.\\
	
	
	\textbf{D1:} Optimiza situaciones de la vida cotidiana mediante la modelación de situaciones con la ecuación cuadrática
	
	\textbf{D2:} Factoriza adecuadamente polinomios algebraicos con base a los diferentes casos de factorización
	
	\textbf{D3:} Encuentra los ceros de un polinomio mediante la factorización.
	
	\textbf{D4:} Divide adecuadamente expresiones algebraicas mediante el proceso análogo a la división de números naturales.
	
	\textbf{D5:} Realiza operaciones como suma y resta de polinomios, encontrando el MCD de polinomios.
	
	\textbf{D6:} Realiza operaciones como suma y resta de polinomios, encontrando el MCM de polinomios.
	
	\textbf{D7:} Reconoce y grafica la función exponencial
	

	
	
	%---------------------------------------------------
	%---------------------------------------------------
	%---------------------------------------------------
	\vspace{1cm}
	
	\begin{itemize}
		\item \textbf{Desempeños 1}
	
		
		\begin{figure}[h]
			\includegraphics[scale=0.4]{ecuacionescuadraticas.png}
		\end{figure}
	
	
	Resueleve los siguientes problemas modelando el problema mediante una ecuación cuadrática.
	
		\begin{figure}[h!]
			\includegraphics[scale=0.32]{problemas.png}
		\end{figure}
	
	
	
	\newpage
	
		\item \textbf{Desempeños 2, 3, 4, 5, 6}
		
		
		
		\begin{figure}[h!]
			\includegraphics[scale=0.5]{mcm.png}
		\end{figure}

		\begin{figure}[h]
			\includegraphics[scale=0.4]{mcd.png}
		\end{figure}		
	
	\newpage
	
		\item \textbf{Desempeño 7}
		

		\begin{figure}[h]
			\includegraphics[scale=0.6]{exp.png}
		\end{figure}

	
	\end{itemize}
	
	
\end{document}